% Please do not change the document class
\documentclass{scrartcl}

% Please do not change these packages
\usepackage[hidelinks]{hyperref}
\usepackage[none]{hyphenat}
\usepackage{setspace}
\doublespace

% You may add additional packages here
\usepackage{amsmath}

% Please include a clear, concise, and descriptive title
\title{Illumination for Computer Generated Pictures}

% Subtitle
\subtitle{COMP110 - Research Journal - Computing}

% Please put your student number in the author field
\author{1707502}

\begin{document}

\maketitle


\section{Introduction}

This Journal will detail my research on Bui Tuong Phong's seminal article "Illumination for computer generated pictures" and its influence on object modeling in the industry today. This article goes into particular detail about Existing shading techniques of the time , the current issues with long rendering time for detailed shaded objects and the distant goal of a real time display system that can generate pictures at 30 frames per second. I will be referencing both new and older articles that connect to these topics within my main piece of research material. 

\section{Shading Techniques}

Phong talks about many different shading techniques that have been created throughout this section of his article. Phong’s first talks about Warnock’s shading and how upon projection onto the cathode-ray tube screen a great amount of depth is lost which makes the images look flat. To increase the amount of depth Warnock decreased the intensity of the reflected light from the object with the distance between the light source and the object.  Next he talked about Newell and sancha who had some ideas on creating transparency and highlights for objects by using not just the highlights created by the incident light source but also the reflection of light from other objects in the scene. 
\cite{Phong}

\section{Conclusion}



\bibliographystyle{ieeetran}
\bibliography{references}


\end{document}